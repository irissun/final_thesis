How to quote: 

\begin{quote}
	Without intending any necessary implication of causality, it is convenient to use the phrase “effect size” to mean “the degree to which the phenomenon is present in the population,” or “the degree to which the null hypothesis is false.”. By the above route it can now readily be clear that when the null hypothesis is false, it is false to some specific degree, i.e., the effect size (ES) is some specific non-zero value in the population. The larger this value, the greater the degree to which the phenomenon under study is manifested. (pp. 9-10)
\end{quote}
 If you quote, you have to quote in this way: ``a named expression that maps data, statistics, or parameters onto a quantity that represents the magnitude of some phenomenon" \cite[p.141]{kelley2012effect}. 
 
Examples of Equations:
\begin{equation}
\delta = \frac{\mu^E - \mu^C}{\sigma},
\label{pop_smd}
\end{equation}
 